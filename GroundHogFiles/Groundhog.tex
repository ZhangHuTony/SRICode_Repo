% Options for packages loaded elsewhere
\PassOptionsToPackage{unicode}{hyperref}
\PassOptionsToPackage{hyphens}{url}
%
\documentclass[
]{article}
\usepackage{amsmath,amssymb}
\usepackage{iftex}
\ifPDFTeX
  \usepackage[T1]{fontenc}
  \usepackage[utf8]{inputenc}
  \usepackage{textcomp} % provide euro and other symbols
\else % if luatex or xetex
  \usepackage{unicode-math} % this also loads fontspec
  \defaultfontfeatures{Scale=MatchLowercase}
  \defaultfontfeatures[\rmfamily]{Ligatures=TeX,Scale=1}
\fi
\usepackage{lmodern}
\ifPDFTeX\else
  % xetex/luatex font selection
\fi
% Use upquote if available, for straight quotes in verbatim environments
\IfFileExists{upquote.sty}{\usepackage{upquote}}{}
\IfFileExists{microtype.sty}{% use microtype if available
  \usepackage[]{microtype}
  \UseMicrotypeSet[protrusion]{basicmath} % disable protrusion for tt fonts
}{}
\makeatletter
\@ifundefined{KOMAClassName}{% if non-KOMA class
  \IfFileExists{parskip.sty}{%
    \usepackage{parskip}
  }{% else
    \setlength{\parindent}{0pt}
    \setlength{\parskip}{6pt plus 2pt minus 1pt}}
}{% if KOMA class
  \KOMAoptions{parskip=half}}
\makeatother
\usepackage{xcolor}
\usepackage[margin=1in]{geometry}
\usepackage{color}
\usepackage{fancyvrb}
\newcommand{\VerbBar}{|}
\newcommand{\VERB}{\Verb[commandchars=\\\{\}]}
\DefineVerbatimEnvironment{Highlighting}{Verbatim}{commandchars=\\\{\}}
% Add ',fontsize=\small' for more characters per line
\usepackage{framed}
\definecolor{shadecolor}{RGB}{248,248,248}
\newenvironment{Shaded}{\begin{snugshade}}{\end{snugshade}}
\newcommand{\AlertTok}[1]{\textcolor[rgb]{0.94,0.16,0.16}{#1}}
\newcommand{\AnnotationTok}[1]{\textcolor[rgb]{0.56,0.35,0.01}{\textbf{\textit{#1}}}}
\newcommand{\AttributeTok}[1]{\textcolor[rgb]{0.13,0.29,0.53}{#1}}
\newcommand{\BaseNTok}[1]{\textcolor[rgb]{0.00,0.00,0.81}{#1}}
\newcommand{\BuiltInTok}[1]{#1}
\newcommand{\CharTok}[1]{\textcolor[rgb]{0.31,0.60,0.02}{#1}}
\newcommand{\CommentTok}[1]{\textcolor[rgb]{0.56,0.35,0.01}{\textit{#1}}}
\newcommand{\CommentVarTok}[1]{\textcolor[rgb]{0.56,0.35,0.01}{\textbf{\textit{#1}}}}
\newcommand{\ConstantTok}[1]{\textcolor[rgb]{0.56,0.35,0.01}{#1}}
\newcommand{\ControlFlowTok}[1]{\textcolor[rgb]{0.13,0.29,0.53}{\textbf{#1}}}
\newcommand{\DataTypeTok}[1]{\textcolor[rgb]{0.13,0.29,0.53}{#1}}
\newcommand{\DecValTok}[1]{\textcolor[rgb]{0.00,0.00,0.81}{#1}}
\newcommand{\DocumentationTok}[1]{\textcolor[rgb]{0.56,0.35,0.01}{\textbf{\textit{#1}}}}
\newcommand{\ErrorTok}[1]{\textcolor[rgb]{0.64,0.00,0.00}{\textbf{#1}}}
\newcommand{\ExtensionTok}[1]{#1}
\newcommand{\FloatTok}[1]{\textcolor[rgb]{0.00,0.00,0.81}{#1}}
\newcommand{\FunctionTok}[1]{\textcolor[rgb]{0.13,0.29,0.53}{\textbf{#1}}}
\newcommand{\ImportTok}[1]{#1}
\newcommand{\InformationTok}[1]{\textcolor[rgb]{0.56,0.35,0.01}{\textbf{\textit{#1}}}}
\newcommand{\KeywordTok}[1]{\textcolor[rgb]{0.13,0.29,0.53}{\textbf{#1}}}
\newcommand{\NormalTok}[1]{#1}
\newcommand{\OperatorTok}[1]{\textcolor[rgb]{0.81,0.36,0.00}{\textbf{#1}}}
\newcommand{\OtherTok}[1]{\textcolor[rgb]{0.56,0.35,0.01}{#1}}
\newcommand{\PreprocessorTok}[1]{\textcolor[rgb]{0.56,0.35,0.01}{\textit{#1}}}
\newcommand{\RegionMarkerTok}[1]{#1}
\newcommand{\SpecialCharTok}[1]{\textcolor[rgb]{0.81,0.36,0.00}{\textbf{#1}}}
\newcommand{\SpecialStringTok}[1]{\textcolor[rgb]{0.31,0.60,0.02}{#1}}
\newcommand{\StringTok}[1]{\textcolor[rgb]{0.31,0.60,0.02}{#1}}
\newcommand{\VariableTok}[1]{\textcolor[rgb]{0.00,0.00,0.00}{#1}}
\newcommand{\VerbatimStringTok}[1]{\textcolor[rgb]{0.31,0.60,0.02}{#1}}
\newcommand{\WarningTok}[1]{\textcolor[rgb]{0.56,0.35,0.01}{\textbf{\textit{#1}}}}
\usepackage{graphicx}
\makeatletter
\def\maxwidth{\ifdim\Gin@nat@width>\linewidth\linewidth\else\Gin@nat@width\fi}
\def\maxheight{\ifdim\Gin@nat@height>\textheight\textheight\else\Gin@nat@height\fi}
\makeatother
% Scale images if necessary, so that they will not overflow the page
% margins by default, and it is still possible to overwrite the defaults
% using explicit options in \includegraphics[width, height, ...]{}
\setkeys{Gin}{width=\maxwidth,height=\maxheight,keepaspectratio}
% Set default figure placement to htbp
\makeatletter
\def\fps@figure{htbp}
\makeatother
\setlength{\emergencystretch}{3em} % prevent overfull lines
\providecommand{\tightlist}{%
  \setlength{\itemsep}{0pt}\setlength{\parskip}{0pt}}
\setcounter{secnumdepth}{-\maxdimen} % remove section numbering
\ifLuaTeX
  \usepackage{selnolig}  % disable illegal ligatures
\fi
\IfFileExists{bookmark.sty}{\usepackage{bookmark}}{\usepackage{hyperref}}
\IfFileExists{xurl.sty}{\usepackage{xurl}}{} % add URL line breaks if available
\urlstyle{same}
\hypersetup{
  pdftitle={Groundhog DTMC},
  pdfauthor={Montana Ferita},
  hidelinks,
  pdfcreator={LaTeX via pandoc}}

\title{Groundhog DTMC}
\author{Montana Ferita}
\date{2024-02-01}

\begin{document}
\maketitle

\textbf{Due Feb 7th @ 11:59pm}

\hypertarget{questions}{%
\section{Questions}\label{questions}}

\textbf{In your own words, describe what a discrete time markov chain
is?} - A stochastic system which state is influenced by the previous
state.

\textbf{Punxsutawney Phil is nervous about his big day and he keeps
practicing going inside and outside of his hole. Suppose the probability
he leaves his hole is 0.4 and the probability he goes back in his hole
is 0.3. What are the conditional probabilities?} - Pr(\(I_t|O_{t-1}\)):
0.4 - Pr(\(O_t|I_{t-1}\)): 0.3

\hypertarget{model-creation}{%
\section{Model Creation}\label{model-creation}}

\textbf{Write the discrete-time dynamical system of equations.} Note to
write an equation, you need to put it in double dollar signs. To create
a subscript, like \[A_{b+c}\] you need to use the curly braces. However,
if you subscript is only one letter, than you only need to use the
underscore, \[A_b\]. The double backslash symbol creates a new line.

\[
O_{t+1} = 0.4
I_{t+1} = 0.3 
\]

\hypertarget{long-term-probability}{%
\section{Long-term Probability}\label{long-term-probability}}

\textbf{Find the long-term probability Punxsutawney is outside.} -
\(O^* = \frac{4}{7}\)

\hypertarget{simulation}{%
\section{Simulation}\label{simulation}}

\textbf{Create a discrete time Markov Chain model of Punxsutawney going
inside and outside of his hole. Suppose the he starts inside and run
this model for 100 time steps. For each line of the code, describe what
the line is doing.}

\begin{Shaded}
\begin{Highlighting}[]
\CommentTok{\#Define number of steps}
\NormalTok{numSteps }\OtherTok{\textless{}{-}} \DecValTok{100}

\CommentTok{\#Create list which will hold states of all steps (inside or outside)}
\NormalTok{statesList }\OtherTok{\textless{}{-}} \FunctionTok{numeric}\NormalTok{(numSteps)}
\CommentTok{\#starts inside}
\NormalTok{statesList[}\DecValTok{1}\NormalTok{] }\OtherTok{=} \DecValTok{1}
\CommentTok{\#1 denotes inside and 2 denotes outside}
\NormalTok{states }\OtherTok{\textless{}{-}} \FunctionTok{c}\NormalTok{(}\DecValTok{1}\NormalTok{,}\DecValTok{2}\NormalTok{)}

\CommentTok{\#Create matrix for all probabilities}
\NormalTok{statesBefore }\OtherTok{\textless{}{-}} \FunctionTok{c}\NormalTok{(}\StringTok{"previously inside"}\NormalTok{, }\StringTok{"previously outside"}\NormalTok{)}
\NormalTok{statesAfter }\OtherTok{\textless{}{-}} \FunctionTok{c}\NormalTok{(}\StringTok{"now inside"}\NormalTok{, }\StringTok{"now outside"}\NormalTok{)}
\NormalTok{probINIB}\OtherTok{\textless{}{-}} \FloatTok{0.6}
\NormalTok{probONIB}\OtherTok{\textless{}{-}}\FloatTok{0.4}
\NormalTok{probINOB}\OtherTok{\textless{}{-}} \FloatTok{0.3}
\NormalTok{probONOB}\OtherTok{\textless{}{-}}\FloatTok{0.7}

\NormalTok{Transition}\OtherTok{\textless{}{-}}\FunctionTok{matrix}\NormalTok{(}\FunctionTok{c}\NormalTok{(probINIB,probINOB,probONIB,probONOB),}\AttributeTok{nrow=}\DecValTok{2}\NormalTok{,}\AttributeTok{dimnames =} \FunctionTok{list}\NormalTok{(statesBefore,statesAfter))}

\CommentTok{\#takes 100 hundred samples, using matrix and previous state as index to choose the probabilites for the states}
\ControlFlowTok{for}\NormalTok{(i }\ControlFlowTok{in} \DecValTok{2}\SpecialCharTok{:}\NormalTok{numSteps)\{}
\NormalTok{  statesList[i] }\OtherTok{=} \FunctionTok{sample}\NormalTok{(states, }\DecValTok{1}\NormalTok{, }\AttributeTok{prob =}\NormalTok{ Transition[statesList[i}\DecValTok{{-}1}\NormalTok{],] )}
\NormalTok{\}}
\end{Highlighting}
\end{Shaded}

\textbf{Now create a graph showing when Punxsutawney goes inside and
outside of his hole. Count the number of times he is outside. Does this
agree with the long-term probability? (Do not count by hand)}

\begin{Shaded}
\begin{Highlighting}[]
\FunctionTok{plot}\NormalTok{(statesList)}
\end{Highlighting}
\end{Shaded}

\includegraphics{Groundhog_files/figure-latex/unnamed-chunk-2-1.pdf}

\begin{Shaded}
\begin{Highlighting}[]
\FunctionTok{print}\NormalTok{(}\FunctionTok{sum}\NormalTok{(statesList }\SpecialCharTok{==} \DecValTok{2}\NormalTok{))}
\end{Highlighting}
\end{Shaded}

\begin{verbatim}
## [1] 56
\end{verbatim}

56 is very close to 4/7 so yes it does agree with th elong term
probability

\textbf{Congrats, you did it! Please submit both your markdown file and
the html file (click on knit-\textgreater knit to html) on Canvas.}

\end{document}
